\documentclass[a4paper,10pt]{article}
\usepackage[english]{babel}
\usepackage[OT1]{fontenc}
\usepackage[utf8]{inputenc}
\usepackage{fancyhdr}
\usepackage{mydoc}
\usepackage[pdftex,colorlinks=true,
				 pdfstartview=FitV,
				 linkcolor=blue,
				 citecolor=blue,
				 urlcolor=blue,
		  ]{hyperref}

\begin{document}
\pagestyle{fancy}
\fancyfoot{}
\fancyhead{}
\renewcommand{\sectionmark}[1]{\markboth{\sf\thesection.\ #1}{}}
\renewcommand{\subsectionmark}[1]{}
\fancyhead[R]{{\rmfamily\thepage}}

\title{CVS to git transition guide}

\subtitle{for ACES2 developers}
\author{{\sf Jonas Jus\'elius}}
\address{
{\sf University of Tromsø}\\
{\sf Department of Chemistry}\\
{\sf N-9037 University of Tromsø, Norway}
}
\years{2007}
%\abstract{Quick-start guide how to get started with git and how to migrate old 
%CVS repositories.}

\maketitle

\section{Preliminaries}
This is not a git manual, nor a proper HOWTO either. There is a lot of decent
documentation available on the web, and all git commands are very well
documented, so there is no need to repeat them here. 
This document explains step-by-step how to move your local changes
in your CVS repositories to the newly created git repository. It assumes that
you have a recent version of git installed on your system, and that you have
familiarised your self with the basic git operations by reading the available
git documentation at \url{http://git.or.cz}. 

The second part of this guide explains some of the differences between git and
CVS, which at first sight might be a bit hard to understand. Although git is
well documented, some of the documentation is somewhat scattered, or seems to
be written for computer scientists.

If you have not done so yet I
recommend you start by reading the following two short tutorials:
\begin{description}
\item[Git for CVS users]:\\
\url{http://www.kernel.org/pub/software/scm/git/docs/cvs-migration.html}
\item[The git tutorial]:\\
\url{http://www.kernel.org/pub/software/scm/git/docs/tutorial.html}
\end{description}
Furthermore, the following documents may be of use:
\begin{description}
\item[Everyday git with 20 commands or so]:\\
\url{http://www.kernel.org/pub/software/scm/git/docs/everyday.html}
\item[Git user's manual]:\\
\url{http://www.kernel.org/pub/software/scm/git/docs/user-maunal.html}
\end{description}

\section{Introduction to git}
\subsection{Branches in git}
\subsection{Useful stuff}

\section{Migrating your CVS repository}

\end{document}

