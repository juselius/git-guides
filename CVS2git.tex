\documentclass[a4paper,10pt]{article}
\usepackage[english]{babel}
\usepackage[OT1]{fontenc}
\usepackage[utf8]{inputenc}
\usepackage{fancyhdr}
\usepackage{mydoc}
\usepackage[pdftex,colorlinks=true,
				 pdfstartview=FitV,
				 linkcolor=blue,
				 citecolor=blue,
				 urlcolor=blue,
		  ]{hyperref}

\begin{document}
\pagestyle{fancy}
\fancyfoot{}
\fancyhead{}
\renewcommand{\sectionmark}[1]{\markboth{\sf\thesection.\ #1}{}}
\renewcommand{\subsectionmark}[1]{}
\fancyhead[R]{{\rmfamily\thepage}}

\title{CVS to git transition guide}

\subtitle{for ACES2 developers}
\author{{\sf Jonas Jus\'elius}}
\address{
{\sf University of Tromsø}\\
{\sf Department of Chemistry}\\
{\sf N-9037 University of Tromsø, Norway}
}
\years{2007}
%\abstract{Quick-start guide how to get started with git and how to migrate old 
%CVS repositories.}

\maketitle

\section{Preliminaries}
This is not a git manual, nor a proper HOWTO either. There is a lot of decent
documentation available on the web, and all git commands are very well
documented, so there is no need to repeat them here. 
This document explains step-by-step how to move your local changes
in your CVS repositories to the newly created git repository. It assumes that
you have a recent version of git installed on your system, and that you have
familiarised your self with the basic git operations by reading the available
git documentation at \url{http://git.or.cz}. 

The second part of this guide explains some of the differences between git and
CVS, which at first sight might be a bit hard to understand. Although git is
well documented, some of the documentation is somewhat scattered, or seems to
be written for computer scientists.

If you have not done so yet I
recommend you start by reading the following two short tutorials:
\begin{description}
\item[Git for CVS users]:\\
\url{http://www.kernel.org/pub/software/scm/git/docs/cvs-migration.html}
\item[The git tutorial]:\\
\url{http://www.kernel.org/pub/software/scm/git/docs/tutorial.html}
\end{description}
Furthermore, the following documents may be of use:
\begin{description}
\item[Everyday git with 20 commands or so]:\\
\url{http://www.kernel.org/pub/software/scm/git/docs/everyday.html}
\item[Git user's manual]:\\
\url{http://www.kernel.org/pub/software/scm/git/docs/user-maunal.html}
\end{description}

\section{Migrating your CVS repository}
\subsection{Before you start}
Before doing anything else, I suggest that you setup some default
configuration variables for git. Git uses a number of config files, system
wide, per user and per repository (see the git-config man page for more info).
As a minimum you should set the following options:
\begin{verbatim}
$ git-config --global user.name "Your Name"
$ git-config --global user.email "my@email.com"
\end{verbatim}
These options are written in ~/.gitconfig, and are used by git to ensure that
your commit messages are sensible, since user names and mail addresses are not
necessarily set properly on all machines. In addition you might want to enable
the following options as well:
\begin{verbatim}
$ git-config --global color.branch auto
$ git-config --global color.status auto
$ git-config --global color.diff false
\end{verbatim}
In order to save space you can also enable compression
\begin{verbatim}
$ git-config --global core.compression 1
$ git-config --global core.loosecompression 1
\end{verbatim}
If you want to use some external (graphical) merge tool to resolve conflicts:
\begin{verbatim}
$ git-config --global merge.tool meld
\end{verbatim}
Meld is a fantastic merge tool, and I strongly suggest you have a look at it.
Other valid possibilities are kdiff3 and xxdiff (amongst others).

\subsection{Updating your CVS repository}
Since the old CVS server is no longer accepting new modifications, you need to
move all of your local modifications under git and then commit them. To do
this you need to follow these directions carefully. Before you start I
strongly suggest that you backup your working copy!
\begin{enumerate}
\item The first thing you must do is to synchronise your working copy with the
CVS server, in order to make sure that your files are in the right state.
Files with local modifications will not be overwritten or updated. 
\end{enumerate}

\section{Introduction to git}

\subsection{Branches in git}
\subsection{Useful stuff}


\end{document}

