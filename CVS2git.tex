\documentclass[a4paper,10pt]{article}
\usepackage[english]{babel}
\usepackage[OT1]{fontenc}
\usepackage[utf8]{inputenc}
\usepackage{fancyhdr}
\usepackage{mydoc}
\usepackage[pdftex,colorlinks=true,
				 pdfstartview=FitV,
				 linkcolor=blue,
				 citecolor=blue,
				 urlcolor=blue,
		  ]{hyperref}

\begin{document}
\pagestyle{fancy}
\fancyfoot{}
\fancyhead{}
\renewcommand{\sectionmark}[1]{\markboth{\sf\thesection.\ #1}{}}
\renewcommand{\subsectionmark}[1]{}
\fancyhead[R]{{\rmfamily\thepage}}

\title{CVS to git transition guide}

\subtitle{for ACES2 developers}
\author{{\sf Jonas Jus\'elius}}
\address{
{\sf University of Tromsø}\\
{\sf Department of Chemistry}\\
{\sf N-9037 University of Tromsø, Norway}
}
\years{2007}
%\abstract{Quick-start guide how to get started with git and how to migrate old 
%CVS repositories.}

\maketitle

\section{Preliminaries}
This is not a git manual, nor a proper HOWTO either. There is a lot of decent
documentation available on the web, and all git commands are very well
documented, so there is no need to repeat them here. 
This document explains step-by-step how to move your local changes
in your CVS repositories to the newly created git repository. It assumes that
you have a recent version of git installed on your system, and that you have
familiarised your self with the basic git operations by reading the available
git documentation at \url{http://git.or.cz}. 

The second part of this guide explains some of the differences between git and
CVS, which at first sight might be a bit hard to understand. Although git is
well documented, some of the documentation is somewhat scattered, or seems to
be written for computer scientists.

If you have not done so yet I
recommend you start by reading the following two short tutorials:
\begin{description}
\item[Git for CVS users]:\\
\url{http://www.kernel.org/pub/software/scm/git/docs/cvs-migration.html}
\item[The git tutorial]:\\
\url{http://www.kernel.org/pub/software/scm/git/docs/tutorial.html}
\end{description}
Furthermore, the following documents may be of use:
\begin{description}
\item[Everyday git with 20 commands or so]:\\
\url{http://www.kernel.org/pub/software/scm/git/docs/everyday.html}
\item[Git user's manual]:\\
\url{http://www.kernel.org/pub/software/scm/git/docs/user-maunal.html}
\end{description}

\section{Migrating your CVS repository}
\subsection{Before you start}
Before doing anything else, I suggest that you setup some default
configuration variables for git. Git uses a number of config files, system
wide, per user and per repository (see the git-config man page for more info).
As a minimum you should set the following options:
\begin{verbatim}
$ git-config --global user.name "Your Name"
$ git-config --global user.email "my@email.com"
\end{verbatim}
These options are written in ~/.gitconfig, and are used by git to ensure that
your commit messages are sensible, since user names and mail addresses are not
necessarily set properly on all machines. In addition you might want to enable
the following options as well:
\begin{verbatim}
$ git-config --global color.branch auto
$ git-config --global color.status auto
$ git-config --global color.diff false
\end{verbatim}
In order to save space you can also enable compression
\begin{verbatim}
$ git-config --global core.compression 1
$ git-config --global core.loosecompression 1
\end{verbatim}
If you want to use some external (graphical) merge tool to resolve conflicts:
\begin{verbatim}
$ git-config --global merge.tool meld
\end{verbatim}
Meld is a fantastic merge tool, and I strongly suggest you have a look at it.
Other valid possibilities are kdiff3 and xxdiff (amongst others).

\subsection{Updating your CVS repository}
Since the old CVS server is no longer accepting new modifications, you need to
move all of your local modifications under git and then commit them. To do
this you need to follow these directions carefully. Before you start I
strongly suggest that you backup your working copy!
\begin{description}
\item[Update] The first thing you must do is to synchronise your working
copy with the CVS server by running
\begin{verbatim}
$ cvs update
\end{verbatim}
This makes sure that your files are in the right state. 
Files with local modifications will not be overwritten nor
updated. If you have modified files that meanwhile have been modified on the
master you will probably have a number of conflicts. You need to resolve these
conflicts before continuing. Edit the files with conflicts and pick the
correct pieces of code between the markers.
\item[Diff] The next step is to create a patch file to migrate all your
changes from CVS to your new git repository. 
Now that all files, except your locally modified files, are in the same state
as on the master you can run
\begin{verbatim}
$ cvs diff -u -N >migration.diff
\end{verbatim}
This file contains all the differences between the master and your working
copy.
\item[Clone] Now you are in a position to clone the new git master repository!
Change your working directory to where you want to create your new repository
and run
\begin{verbatim}
$ git clone user@feynman.chemie.uni-mainz.de:/mnt/software/git/aces2.git
$ cd aces2 
\end{verbatim}
Optionally you can also specify the name of the new repository.
\item[Branch and checkout] 
Before you do anything else you need to create a new (local) branch which is
in the same state as the old master CVS, and switch to the new branch. The new
branch will be called migration (or whatever) and must be created from the tag
CVS\_migration\_HEAD:
\begin{verbatim}
$ git branch migration CVS_migration_HEAD
$ git checkout migration
\end{verbatim}
\item[Patch] You are now ready to apply the patch you created and bring your
own local modifications into your git repository
\begin{verbatim}
$ patch -p0 < /path/to/old/aces2/migration.diff
\end{verbatim}
The patch should apply without a hitch if you have done everything correctly.
You can now run 
\begin{verbatim}
$ git status
\end{verbatim}
to show the status of your repository (i.e. modified files, deleted files,
files that needs to be added, etc.).
\end{description}

Congratulations! You are now in the same situation as before the switch!

Before continuing you probably want to read the rest of this document, and in
particular the section on commits and patch sets. Before you add files to
commit you should probably think closely about how changes are logically
related and how they could be grouped into sets, instead of committing
everything as one big blob. It's better to have many, many small commits than
a few big ones, since this helps picking out particular patch sets and
applying them to other branches (this is known as cherry picking).
\begin{description}
\item[Commit] The following step in the migration process is to commit all
your changes in your local repository. Don't worry, you will not mess up
anything on the master, or even make your changes public yet. This is a
personal commit so far. There is a convenient tool to pick, group and commit 
your changes
\begin{verbatim}
$ git citool
\end{verbatim}
If you rather do everything by hand you can either do a
\begin{verbatim}
$ git commit -a
\end{verbatim}
to do what you probably should not, i.e. commit everything as one huge commit.
Then run 
\begin{verbatim}
$ git status
\end{verbatim}
to check that you have not missed anything. If you have untracked files you
need to add them, and commit them (you can also do this before you commit the
first time)
\begin{verbatim}
$ git add file(s)...
$ git commit
\end{verbatim}

If you want to group your changes into logical sets, you should repeatedly run
\begin{verbatim}
$ git add file(s) you want to commit
$ git commit
\end{verbatim}
until all changes have been committed.
\end{description}
\section{Introduction to git}

\subsection{Branches in git}
\subsection{Useful stuff}


\end{document}

